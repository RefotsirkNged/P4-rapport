\section{Scanner and Parser phase}

The scanner (also known as the lexer) and parser used for \lang{} has been generated using SableCC, with a modified version of the grammar shown in Section \ref{sec:OurSyntax}. The actual code used with SableCC can be found in \ref{FGramma}. By running code in \lang{} through the parser, an Abstract Syntax Tree (AST) is generated, which will then be used in the semantic analysis phase, and in the code-generation phase.
The grammar definition is split into two parts. The first part defines the syntax of \lang{} and aids the creation of a CST structure, and the second part defines an AST structure for input generated by the first phase of the compiler. The CST is used internally by the code generated by SableCC, and is never touched in the implementation of \lang{}. The AST, however, is the result the syntax analysis phase produces, and it is what is used further on in the following phases. 
The grammar for the AST and the CST are very similar. The main difference is that the AST grammar can be ambigious, whereas the CST grammar must be unambigous. Both grammars can be found in the appendix, AST grammar at appendix \ref{AST}, and the CST at the top of appendix \ref{FGramma}. They are both written in the style required by SableCC, although still close to the original grammar.
\subsection{Implementation of Syntax in \lang{}}
The implementation of \lang{}'s syntax is handled entirely by SableCC, as it generates both the lexer and the parser. There is no custom code in this part of the compiler. It is all generated by SableCC. The only "custom" code is an object instance which contains the AST after being produced by the lexer/parser. The code used for calling the parser and lexer is implemented as follows:
\begin{figure}[H]
    \centering
    
    \begin{lstlisting}[style=gglang]
        PushbackReader pushbackReader = new PushbackReader(new FileReader(addLibrary(file)));
        Parser parser = new Parser(new Lexer(pushbackReader));
        Start tree = parser.parse();
    \end{lstlisting}
    \caption{The code for calling the parser and lexer.\label{fig:parslex}}
\end{figure}
"PushbackReader" is the lexer, and "Parser" is the parser. "tree" is the object instance containing the AST after the parser is done. 