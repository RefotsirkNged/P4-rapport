\chapter{Introduction}
\begin{comment}
1. To whom is this a problem? (i.e., who is going to benefit from your solution?)
Society in general. According to http://www.sitepoint.com/6-reasons-why-your-child-and-mine-should-learn-to-code, there will be 1 million unfilled jobs in programming/IT by 2020 if nothing is done.

2. Why is this a problem? (motivate why you believe this is an interesting problem)
It is a problem because the IT industry is growing rapidly, and has been ever since the invention of computers. The amount of people getting an education related to programming is not enough to cover the needs of the world. One way to remedy this, is to teach children programming at an early age, or at least to get them interested in it. We already have things like Scratch, but the step from that and into a real programming language like Java or C\# can be a daunting challenge for many kids. Therefore, we think it could be interesting to develop a language that lands somewhere in between Scratch and a real language. 

3. How do you plan to solve this problem? (sketch a preliminary problem Analysis, listing the techniques you think will be used in this project)
We plan to take a look at different programming languages for kids that already exist, like Scratch, and see if we can hit an abstraction level somewhere in between tools like Scratch, and a real programming language. We want to focus the language on creating simple 2D games, as this will help getting more children interested in programming. 
We will implement the language by compiling it to Java, so that we can use the game library Slick2D.


In our modern society, everything is controlled by computers. Most people, in the western world, interact directly with a computer every day, and many even carry one with them, namely their smartphone. However, as technology becomes a more integrated part of our lives, more and more abstraction layers are created to distinct the user from the 'machinery'.

The consequences of this is that there is a much greater need to understand, not only how to use these devices, but also how to utilise them in new and innovative ways. Furthermore, the IT industry has been growing at increasingly faster rates, and nothing indicates that it is going to slow down any time soon. It is therefore important to educate more people in the field of computer science.

A method to reach this, is to try and invoke an interest in the field early on, by introducing children to programming. Since a lot of children have an interest in video games, we believe that this can help to develop this wanted interest, by giving them the opportunity to create their own video games with basic tools. Furthermore, it is more motivating to be able to create graphics and make it react to what you type, rather than just seeing some text in a terminal window.








it is not needed to be able to write code but just the way of thinking behind it. This can give a lot of benefits to children and adults in the way they approach a problem, where they have a big problem that they keep on making into smaller and smaller problems so they end up with having a lot of small and easy-to-understand problems, or the way they work around a problem with lot more logical approach. 

\end{comment}




In our modern society, everything is controlled by computers. Most people, in the western world, interact directly with a computer every day, and many even carry one with them, namely their smartphone\citep{geny}. However, as technology becomes a more integrated part of the everyday life, more and more abstraction layers are created to distinct the user from the inner workings of the technology.

The consequences of this is that there is a much greater need to understand, not only how to use these devices, but also how to utilise them in new and innovative ways. Furthermore, the IT industry has been growing at increasingly faster rates and nothing indicates that it is going to slow down any time soon \citep{BLSGrowth}. It is therefore important to educate more people in the field of computer science.

A way to achieve this, is to try and invoke an interest in the field early on, by introducing young people to programming. Since a lot of young people have an interest in video games, this would be a good approach to try and invoke this interest. An example of people doing this is found on Code.org\citep{codeorg}. This company believes that teaching coding via games can help children to develop an interest in programming, by giving them the opportunity to create their own video games using basic tools. Furthermore, it can be more motivating to be able to create graphics and make it react to user input, rather than just seeing some text in a terminal window.

Even though only some will develop an interest for computer science, having the ability to understand how programs work can be very beneficial. The goal is not to teach people all the necessary skills to make a whole program, but to introduce them to the logical thinking that is at the core of programming and writing code in general. \citep{Sitepoint2015}


Building on this, in this report, a programming language will be designed, for use in an educational context.