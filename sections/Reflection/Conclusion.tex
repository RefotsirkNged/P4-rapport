\section{Conclusion}
The goal for this project, as described in the problem statement, was to 

\textit{"Design and develop a beginner friendly programming language that can help to motivate and introduce young people to programming"}

The target audience of the language is young people, who have only had a limited introduction to programming and the different constructs commonly used in programming languages. This is seen in the final design of the language through the different constructs used, such as "downto" and "upto" and by the fact that the language makes use of as few symbols as possible, and instead replaces them with words, such as \textit{begin} and \textit{end}. These design choices are meant to help young people learning programming because they can intuitively understand them. \newline
Another issue was that beginners do not have any experience with IDEs, so a simple GUI was created to allow the user to use the compiler without having to learn how to use complicated IDEs.\newline
The language was made object-oriented, as this is one of the most common programming paradigms used in industries, and since the goal was to introduce them to more common programming structures, object-oriented was a natural choice. It is also easier for the user to map an OOP program to the real world, and vice versa, than it is to map a purely imperative program.\newline
The language focused on having higher readability, rather than focusing on the maintainability and the writability of the language. Readability was the main focus because of the target audience. It was the main goal to make it easier to use and understand what is happening in a piece of code. Writability, especially, suffered because of this, as the high readability imposed a lot of "filler" words in the language in places, where it could have been written with a couple of symbols in many other languages. Maintainability suffered because it takes more work to maintain something in \lang{} due to the added words, but also due to the fact that all classes are written in the same source-file.\newline
The compiler was tested in 2 ways; blackbox and whitebox. Whitebox was done via the test project that was created alongside the compiler project, and it tested if the output of the compiler was correct by comparing it to some expected output. This test covered a wide aspect of the custom constructs used in the language, and as such, it was a good basis for checking whether or not the compiler performed as expected.
Blackbox testing was performed by making a couple of games in the language, and checking what the final game looked like. These are of course not as precise a test as the whitebox tests, but they served as a good indicator for whether or not actual game-making was feasible in \lang{}, which it was. With a very small amount of code, a Ping Pong and a Snake game was made, proving that the language indeed is usable for the intended purpose. 

In the MoSCoW analysis, all the must-have requirements and should-have are implemented. In the could-have category, only the mathematical functions are implemented. Lastly, in the would-have categories, none of the listed features have been implemented. 

The project can be considered successful in the sense that it achieved the goals posed by the problem statement to a sufficient degree of satisfaction.  It is possible to make games using \lang{}, the language itself is easy to read, it is easy to intuitively understand what is going on in a given line of code, it can produce actual games, and it contains constructs that help beginners, who do not have much of an understanding of programming, learn these new constructs. It also met the user-friendliness goal by having a very simple GUI to simplify compilation, and also by not having any requirements in terms of IDE/text editor choice. Of course it can be expanded upon, but as is, the goals of the project were achieved.