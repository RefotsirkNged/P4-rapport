\pdfbookmark[0]{English title page}{label:titlepage_en}
\aautitlepage{%
  \englishprojectinfo{
    Design and implementation of BFGL %title
  }{%
    Design, definition and implementation of programming languages %theme
  }{%
    Spring Semester 2016 %project period
  }{%
    sw405f16 % project group
  }{%
    %list of group members
August Malling Kørvell\\
Carsten Ibsen\\
Mathias Leding\\
Kristoffer Mathiasen Degn\\
  }{%
    %list of supervisors
   Giovanni Bacci 
  }{%
    1 % number of printed copies
  }{%
    \today % date of completion
  }%
}{%department and address
  \textbf{Selma Lagerlöfs Vej}\\
9220 Aalborg Ø\\
Denmark\\
  \href{http://cs.aau.dk}{http://cs.aau.dk}
}{% the abstract
The computer science industry is growing rapidly, and to be able to meet the increased demand for computer scientists and engineers alike, it is essential that the numbers of students within the field grows at an equally increasing rate. A way to try and accomplish this, is to spark an interest in computers and programming from an early age. In this report, the authors tries to do this by designing and implementing a programming language designed to teach young people about programming through the development of something they know and care about: Video games.

The language is designed to be as beginner friendly as possible. An example of how this is done, is that the language design aims to replace symbols with words. To make the user experience focused on actually writing code, as opposed to learning how to use a complicated IDE, a simple graphical user interface has been developed for the \lang{} compiler.

A language specification was written and a corresponding compiler was made.
}

