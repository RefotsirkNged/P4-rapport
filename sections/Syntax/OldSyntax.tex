\chapter{Syntax of \lang{}}
This chapter describes the different kinds of parser generators and analyses what features they have, to be able to make a conclusion for which one is the most suitable for this project. Afterwards, a language for describing syntax is presented, and then the syntax and grammar of \lang{} is presented and explained. 
\section{Parser Generator}\label{sec:pgen}
This is a brief review of the different parser generators. The five different parser generators, covered in this section, are:

\begin{itemize}
    \item Gold Parser
    \item XText
    \item Java Cup
    \item SableCC
    \item ANTLR
\end{itemize}
These were chosen, as they were prominent, and together cover the different sorts of parser generators.

\subsection{Gold Parser}
The Gold parser uses a grammar that is based on the Backus-Naur form(BNF) and regular expressions. Gold Parser is a LALR(1) parser. LALR(1) is simpler than LR(1), while targeting almost the same languages, which makes it easier to use. 
On the downside, Gold Parser has less extensive documentation compared to other popular parsers, making it harder to get started with Gold Parser.

\subsection{XText}
XText uses the grammar Extended Backus-Naur Form(EBNF) and is an LL(1). It implements features integrated into the eclipse IDE \cite{eclipse}, which lets the IDE make highlighting and static syntax checks. These features make the created language more accessible to the end user. Furthermore, XText is well defined through its documentation. XText makes use of ANTLR3 to construct the parse tree, as this feature is not inherent to XText.

\subsection{Java CUP}
Java CUP uses a modified version of EBNF. It is an LR(1) parser and implements a visualization of the  Abstract Syntax Tree(AST). Java CUP is no longer officially supported, but the community has taken over the project and is supporting it under the name Community Z Tools.

\subsection{SableCC}
SableCC uses EBNF for its grammar. it uses, like Gold Parser, LALR(1) for its parser. SableCC has some good features such as an Automatic AST builder for multi-pass compilers, the ability to access sub-nodes by their name instead of their position and a way to pretty print the AST. It also features a clear distinction between user code and machine generated code and is easy to debug. In Version 3.0 it allows declarative grammar transformations.

\subsection{ANTLR4}
ANTLR4 uses the EBNF for its grammar and LL(*)  for its parser. It also has a built-in lexer which XText needs to work. It allows the user to pretty print the AST. ANTLR was first introduced in 1989 and its latest version is ANTLR4. It is used in a wide variety of applications from data warehouses to Twitter.

\subsection{Conclusion}
In this project, it is concluded that SableCC is the best choice for a parser. The reason for this, is the fact that SableCC is a LALR parser, object-oriented, easy to debug, the fact that grammar and action code are separated, has a built-in scanner, can build abstract syntax trees and is in Extended Backus-Naur Form.
An LL(*) parser could have been used in the implementation of \lang{}, since the \lang{} language, and many others, are LL(*).
An LL(*) is simpler than their LR or LALR counterpart, and can be made recursively. \lang{} is however not expected to grow so big that a simpler parser is needed, which is why SableCC is still the chosen parser generator. The grammar would also have to be defined as LL(*) instead of LALR, which is not always easy to do.

\subsection{EBNF notation}
In parser generators like SableCC, the syntax definition is split into several parts. The first part is a token definition section. In this section, common tokens that will be used in the actual definition of the syntax are defined. Usually, the section looks something like this: 

\begin{figure}[H]
    \centering
    \begin{lstlisting}[]
    ...
    mod = '%';
    lparen = '(';
    rparen = ')';
    numval = '-'? digit*;
    newline = (cr | lf | cr lf)+;
    id = letter(letter|digit)*;
    textval = '"' [any - '"']* '"';
    blank = space*;
    comment = '/' '*'[any - ['*' + '/']]* '*' '/';
    \end{lstlisting}
    \caption{Snippet from the Tokens section of the grammar of \lang{} in EBNF}
    \label{fig:EBNFtoken}
\end{figure}

This section mainly exists to unclutter the syntax definition that follows. 
The syntax definition consists of a collection of productions. A production is a left-hand side(LHS) which can be transformed into a right-hand side(RHS). These productions are also called rewriting rules.

The "|" is a way of writing alternatives in a production, so that it can be rewritten as multiple things. It is written like this:

\begin{figure}[H]
    \centering
    \begin{lstlisting}[]
    ...
    type    = num
            | bool
            | text
            | id;
    ...
    \end{lstlisting}
    \caption{Snippet from the Productions section of the grammar of \lang{} in EBNF}
    \label{fig:EBNFalt}
\end{figure}


The LHS is always a nonterminal. A nonterminal is a syntactic part that can be transformed into a terminal symbol by being put through a series of productions.
A terminal symbol is the final point of a production. For example, a "val" nonterminal, denoting a value of some kind, might end up being evaluated into a number(num) or string("text") in \lang{}. 



\section{The syntax of \lang{}}
\label{sec:OurSyntax}
In this section, the grammar of \lang{} is described and arguments are given for the decisions taken. Furthermore, some of the more interesting parts of the grammar are illuminated.
The full grammar, stripped of all the implementation specific parts:\\

    prog        =   global* maindcl newline? classdcl*

    global      =   vardcl newline

    maindcl     =   main begin newline stmt* end

    classdcl    =   newline eclass id inherit? begin newline classbody* end

    classbody   =   vardcl newline                                                            
                |   eventdcl newline                                                
                |   dcl func id lparen formalparam? rparen begin funcbody end newline

    funcbody    =   newline stmt* return

    return      =   {returnid} treturn bexpr newline
                |   {empty}                                                 


    stmt        =   vardcl newline                      
                |   set id to bexpr newline               
                |   for id upto bexpr do [a]:newline stmt* end [b]:newline 
                |   for id downto bexpr do [a]:newline stmt* end [b]:newline
                |   while bexpr do [a]:newline stmt* end [b]:newline  
                |   funcdotcall newline         
                |   ifstmt newline                      

    vardcl      =   dcl type id         
                |   dcl type id to bexpr      
                |   dcl list of type id      

    ifstmt      =   if bexpr then newline stmt* elsestmt? end

    elsestmt    =   else newline stmt*
                |   {elseif} else if bexpr then newline stmt* elsestmt?


    bexpr       =   bexpr or bterm             
                |   bterm                   

    bterm       =   bterm and relation 
                |   relation

    relation    =   expression equals expression
                |   expression notequals expression
                |   expression greater expression
                |   expression less expression       
                |   expression greaterequals expression 
                |   expression lessequals expression   
                |   expression 

    expression  =   expression minus term
                |   expression plus term
                |   term


    term        =   term divide [right]:factor  
                |   term mult [right]:factor  
                |   term mod [right]:factor  
                |   unary               

    unary       =   minus factor 
                |   not factor
                |   factor  

    factor      =   val
                |   id
                |   funcdotcall
                |   iddotcall
                |   lparen bexpr rparen

    type        =   num  
                |   bool     
                |   text                        
                |   id   

    val         =   numval
                |   textval
                |   boolval
                |   this
                |   new id lparen actualparam? rparen

    funccall    =   id lparen actualparam? rparen


    actualparam =   bexpr
                |   bexpr comma actualparam

    formalparam {-> param*}  =   {singleparam} type id                                  {-> [New param.formal(type.type, id)]}
                            |   {mulparam} type id comma formalparam                    {-> [New param.formal(type.type, id), formalparam.param]};

    funcdotcall {-> call}   =   {mulcall} singlecall dot multicallfunc                  {-> New call.dot(singlecall.call, [multicallfunc.call])}
                            |   {single} funccall                                       {-> funccall.call};

    iddotcall {-> call}     =   {mulcall} singlecall dot multicallid                    {-> New call.dot(singlecall.call, [multicallid.call])};

    singlecall  {-> call}   =   {idcall} id                                             {-> New call.var(id)}
                            |   {funccall} funccall                                     {-> funccall.call};

    multicallid {-> call*}  =   {single} id                                             {-> [New call.var(id)]}
                            |   {multi} singlecall dot [rest]:multicallid                 {-> [singlecall.call, rest.call]};

    multicallfunc   {-> call*}  =   {single} funccall                                   {-> [funccall.call]}
                                |   {multi} singlecall dot [rest]:multicallfunc             {-> [singlecall.call, rest.call]};


    inherit     {-> inherit}    =    is type                                            {-> New inherit(type)};

    eventdcl    {-> pdcl}   =   id lparen formalparam* rparen do newline baseconstr? [body]:stmt* end          {-> New pdcl.event(id, [formalparam.param], baseconstr.base, [body.stmt])};

    baseconstr  {-> base}   =   tbase lparen actualparam? rparen newline                {-> New base.base([actualparam.expr])};



As mentioned in section \ref{sec:EBNFinSable}, the initial non-terminal production is the first non-terminal specified in the grammar. The initial non-terminal symbol and its production is called \textit{prog}, and is the entry point of a program in \lang{}:
\begin{figure}[H]
   \centering
    \begin{lstlisting}[]
    ...
    prog = global* maindcl newline? classdcl*
    ...
    \end{lstlisting}
    \caption{Snippet from Pretty Grammar\label{fig:StartProg}}
\end{figure}
As seen in figure \ref{fig:StartProg}, the \textit{prog} consists of an optional series of global variables, the main declaration and an optional series of class declarations.

\lang{} uses newline as line endings. Specifically, it is implemented by using a token specified in the grammar, and having that token trail behind nearly all productions. This is used instead of ";" like in c\# or other c-like languages. 

\subsection{Operator Precedence}
In \lang{}, operator precedence levels are implemented as follows:

\begin{table}[H]
\centering

\begin{tabular}{|l|l|c|}
\hline
  & Name                                                                                                                                  & Symbol                                                                                                 \\ \hline
1 & Parentheses                                                                                                                           & ( )                                                                                                    \\ \hline
2 & \begin{tabular}[c]{@{}l@{}}Unary minus\\ Logical Not\end{tabular}                                                                     & \begin{tabular}[c]{@{}l@{}}-\\ not\end{tabular}                                                        \\ \hline
3 & \begin{tabular}[c]{@{}l@{}}Division\\ Multiplication\\ Modulo\end{tabular}                                                            & \begin{tabular}[c]{@{}l@{}}/\\ *\\ \%\end{tabular}                                                     \\ \hline
4 & \begin{tabular}[c]{@{}l@{}}Addition\\ Subtraction\end{tabular}                                                                        & \begin{tabular}[c]{@{}l@{}}+\\ -\end{tabular}                                                          \\ \hline
5 & \begin{tabular}[c]{@{}l@{}}Equals\\ Not equals\\ Greater than\\ Less than\\ Greater than or equals\\ Less than or equals\end{tabular} & \begin{tabular}[c]{@{}l@{}}=\\ !=\\ \textgreater\\ \textless\\ \textgreater=\\ \textless=\end{tabular} \\ \hline
6 & Logical and                                                                                                                           & and                                                                                                    \\ \hline
7 & Logical or                                                                                                                            & or                                                                                                     \\ \hline
\end{tabular}
\caption{The operator precedence of \lang{} \label{tab:precedence}}
\end{table}

The precedence can be seen in table \ref{tab:precedence}, where the first entry has the highest precedence and the last has the lowest.

The precedence of \lang{} follows the ordinary mathematical rules as described in \citep{precedence}.


\todo{Insert code snippet that shows how this precedence is implemented and describe it}

\subsection{\textit{if}-statements}
\textit{if}-statements in \lang{} always have an associated \textit{end} to eliminate all occurrences of dangling else problems.
As seen in figure \ref{fig:danglingElse}, the dangling else problem does not occur, as the \textit{if}-statement on line 5 ends on line 7, thus can not be associated to the \textit{else} on line 8.

\begin{figure}[H]
    \centering
    \begin{lstlisting}[style=gglang]
    bool someStmt
    bool someOtherStmt
    
    if someStmt then
        if someOtherStmt then
            /* Do something */
        end
    else
        /*Do something */
    end
    \end{lstlisting}
    \caption{Snippet from Pretty Grammar \label{fig:danglingElse}}
\end{figure}

The way \textit{if}-statements are implemented in \lang{} can be seen on figure \ref{fig:ifGrammar}.

\begin{figure}[H]
    \centering
    \begin{lstlisting}[]
    ...
    ifstmt      = "if" bexpr "then" newline stmt* elsestmt? "end"
    
    elsestmt    = "else" newline stmt*
                | "else if" bexpr "then" newline stmt* elsestmt?
    ...
    \end{lstlisting}
    \caption{Snippet from Pretty Grammar\label{fig:ifGrammar}}
\end{figure}

As seen here, \textit{if}-statements consist of the word "\textit{if}", a boolean expression, the word "\textit{then}", a newline, a series of optional statements, an optional else-statement, and the word "\textit{end}", meaning the current \textit{if}-statement has ended.

\subsection{Dot-notations}
The dot notation grammar can be seen in figure \ref{fig:DotNotion}. As seen here, it is possible to make as many multicall \todo{The concept of 'multicall' is not standard. forklar det eller saadan noget} the user wants.

\begin{figure}[H]
    \centering
    \begin{lstlisting}[]
    ...
    classcall   = "." multicall      
                | funccall
                
    singlecall  = id
                | funccall
                
    multicall   = singlecall
                | singlecall "." multicall
    ...
    \end{lstlisting}
    \caption{Snippet from Pretty Grammar\label{fig:DotNotion}}
\end{figure}
\todo{Koden er vist foraeldet. Opdater og saadan}

\todo{Her har gio lavet sin egen udgave af det ovenfor: vv}
\begin{comment}
Genfunccal = (genID'.')* funccal
           | (genfunccal '.')* funccal      Generalized function call
           
           
genId      = (genId '.')* id
           | (genfunccall '.')* id   Generalized variable / class identificator.
\end{comment}

This means that the user can create as many nested dot notations they need.

By being able to call multicall as many times the user wants, they can go between as many refs in class as they want and in the end call a function or a variable in the form of an id or funccall in singlecall. \todo{subsection not clear}



\todo{Suggestion for this section:}
\begin{comment}
To make the description of the syntax of BFGL more clear, Gio suggest to start from section 4.3 (The syntax of BFGL) and give the ENTIRE EBNF for BFGL (replacing trivial token with their definition).


After that you can explain some design choices made in BFGL by referring to the corresponding productions as you did starting from section 4.3.1
\end{comment}
\section {Formal grammar specification of \lang{}}
In \ref{pgen} SableCC was chosen as the parser/scanner generator to be used in this project. Because of this, the grammar has certain traits/naming conventions that had to be enforced for it to work with SableCC. An example of this is the "prog" production:

\begin{figure}[H]
  \centering
    \begin{lstlisting}
     prog        {-> prog}   =   global* maindcl newline? classdcl*
    \end{lstlisting}
  \caption{The prog production. This tells SableCC how to name certain parts of the output.}
  \label{fig:prog}
\end{figure}

In the scope of this chapter the grammar will however have these parts stripped out, to facilitate reading. The full grammar, with nothing stripped out, can be found in appendix \ref{FullGrammar}. \todo{Present the grammar as an EBNF one and just mention that the actual grammar spec for SableCC can be found in appendix \ref{FullGrammar}}


\subsubsection{Operator precedence}
This enforces precedence, because the compiler uses depth-first as the method to travel the AST, as explained in \todo{insert ref}.\todo{<This is wrong apparently. Also not relevant I guess.} \todo{LR parsers (which are bottom-up) have operators with higher precedence deeper in the AST} 



This construction is also used to make nested expressions, because at some point in the derivation of bexpr, it is allowed to derivate backwards to "bexpr".

\subsubsection{Function declarations, Event declarations, Inheritance}
Function calls are achieved with parameter by using "bexpr", which also lets the function take expressions as parameters. Event declarations look a lot like a function call, but the important bit is that it is named differently on the AST. This allows for proper handling of these events later on, seperate from the handling of the function declarations. This is specifically important in \lang{} because it has some build in events that are used to handle game-logic and graphics, which needs to be handled seperately from a normal function declaration.
\lang{} is an OOP language\todo{insert thingy with short explain of oop}, and as such it needs some form of inheritance. This is achieved with the "inherit" production:
\todo{insert inherit prod}
This allows for single-inheritance, not multiple inheritance, as the "type" result of deriving "inherit" can only go to an object, not multiple objects.
\todo{what is the message that you want the reader to get here ?}


\todo{Operator precedence is a way to have unambigous binary (infix) operators without requiring one to write too many parenthesis. It has NOTHING to do with traversal strategy of the AST}



