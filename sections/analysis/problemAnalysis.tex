\newpage
\section{Problem statement}
\label{sec:problemstatement}
The goal of this paper is to design and develop a beginner friendly programming language that can help to motivate and introduce young people to programming.

To build upon the knowledge that the end user has about mathematics and logic, as described in \ref{sub:TargetAudience}, and to give them a foothold in the world of programming, some concepts are more appropriate than others, and will therefore be the main focus of the language described in this paper. These concepts are as follows:

\begin{itemize}
    \item Control flow.\\
    This is important for new programmers to understand, since it lies at the root of the most popular programming languages. Having an understanding of control flow, such as if/else constructs can help the user to better understand structured programming.
    
    \item Variables.\\
    As stated in \ref{sub:TargetAudience}, the end user of the language is assumed to have a basic understanding of algebra. This includes simple algebraic problems such as "Find x, when x + 2 = 4". The next natural step, for the user, is to learn about variables, since this may help them to better understand algebra by giving a real life use for it.
    
    \item Object oriented programming.\\
    Having focus on simple object oriented concepts may help the user to understand the mindset of object oriented programming, making it easier to learn a more complex object oriented programming language in the future.
    Furthermore, it is easier to create games with an object oriented language, and it is easier for a beginner to model the program after real life.
    %Having focus on simple object oriented concepts, may help the user to figure out how to write a program, since this can help the user to model the program after what they see in the real world.
\end{itemize}


Since the language described in this paper is designed specifically to aid beginners with developing simple video games, it is advantageous for the language to be domain specific, i.e a domain-specific language (DSL). This way, the end user can focus on developing and designing games instead of worrying about drawing graphics and other technical problems in game design. To attain this, the language should implement common constructs used in game design to apply abstraction layers to the technical part of developing a video game.

From all this, and based on the rest of the analysis, the following problem statement has been defined:

\textit{How can a domain-specific language that can help to motivate and introduce young people to programming, in particular to concepts like control flow, variables and other basic concepts from object oriented programming, be designed and developed, such that the end user can focus on designing and implementing simple games and not worry about the more technical part of game development?}